\documentclass[conf]{new-aiaa}
\usepackage[utf8]{inputenc}
\usepackage{graphicx}
\usepackage{amsmath}
\usepackage[version=4]{mhchem}
\usepackage{siunitx}
\usepackage{longtable,tabularx}
\setlength\LTleft{0pt} 
\usepackage{amsmath} %math symbols and equations
\usepackage{booktabs} %standard table support
\usepackage{graphicx} %include figures/images
\usepackage{float} %positioning of figures
\usepackage{multicol}  %able to use columns
\usepackage{multirow}
\usepackage{tikz} %create graphics
\usepackage{pgfplots} %create graphics
\usepackage{lettrine} % make a fancy letter
\usepackage{subcaption}  %used to make subfigures

\newcommand\yesnumber{\addtocounter{equation}{1}\tag{\theequation}}



\title{Advancements in Aerodynamic Shape Optimization with Hybrid Laminar Flow Control for Infinite Swept and Finite Wings at Cruise Conditions}

\author{Justin M.\ Pascual \footnote{PhD Candidate, University of Toronto Institute for Aerospace Studies, AIAA Student Member, justin.pascual@mail.utoronto.ca} and David W.\ Zingg \footnote{Distinguished Professor of Computational Aerodynamics and Sustainable Aviation, University of Toronto Institute for Aerospace Studies, AIAA Associate Fellow, david.zingg@utoronto.ca}}
\affil{Institute for Aerospace Studies, University of Toronto, ON M3H 5T6}

\begin{document}

\maketitle

\begin{abstract}

Hybrid laminar flow control (HLFC) offers a promising approach to reducing drag by extending laminar flow regions on wings through the combination of passive shaping and active boundary-layer suction. In this work, a suction boundary condition is integrated into a Reynolds-averaged Navier–Stokes aerodynamic shape optimization framework coupled with the SA-sLM2015cc local correlation-based transition model. The implementation is first validated against benchmark cases, including flat-plate flow and NACA 64A010 airfoil experiments, demonstrating accurate prediction of laminar-to-turbulent transition under varying suction intensities. Lift-constrained drag minimization is then performed with natural laminar flow, i.e,. without suction, for infinite swept wings under estimated Airbus A340 operating conditions. Applying suction upstream of the transition location on the optimized geometries further delays crossflow-dominated transition, yielding drag reductions of up to 15\% relative to fully-turbulent optimized geometries and exceeding reductions achieved by natural laminar flow optimization alone.

\end{abstract}

\section*{Nomenclature}

{\renewcommand\arraystretch{1.0}
\noindent\begin{longtable*}{@{}l @{\quad \quad} l@{}}
$A$ & airfoil area \\
$b$ & span \\
$c$ & chord \\
$C_D$ & coefficient of drag \\
$C_L$ & coefficient of lift \\
$C_Q$ & coefficient of suction \\
$e$ & total energy per unit volume \\
$j$ & mass flux \\
$l$ & length of suction zone for a flat plate \\
$\dot{m}$ & mass flow rate per unit area \\
$M$ & Mach number \\
$Q_{\text{suc}}$ & total volumetric flow rate through the surface \\
$\mathbf{Q_{\text{target}}}$ & solution vector of the conservative flow variables at an individual node \\
$Re$ & Reynolds number \\
$v_{\text{suc}}$ & suction speed \\
$v_{\text{w}}$ & normal component of velocity \\
$U_\infty$ & free-stream velocity \\
$\Lambda$ & sweep angle \\
$\mu_\infty$ & free-stream dynamic viscosity \\
$\rho$ & density \\
$\rho_w$ & local density at the wall \\
$\rho_\infty$ & free-stream density \\
$\xi$ & curvilinear coordinate \\
\end{longtable*}}

\section{Introduction} \label{sec:introduction}

The commercial aviation industry has experienced significant growth over the past few decades. This growing demand has heightened environmental concerns, driving interest in the development of more efficient aircraft designs. Aerodynamic shape optimization supports this effort by iteratively simulating and refining designs to improve specific performance metrics, offering a more systematic and effective alternative to traditional manual design iteration methods.

For a typical commercial aircraft, viscous drag contributes roughly 50\% of the overall drag while operating at cruise conditions \cite{bushnell_overview_1992}. As flow passes over a surface, it transitions from laminar to turbulent, with turbulent flow having higher frictional forces and a thicker boundary layer, increasing the overall drag. There are two main mechanisms through which flow transitions on commercial aircraft, the first being Tollmien-Schlichting (TS) instabilities, which occur in two-dimensional boundary layers in the form of vortices aligned in the spanwise direction. These instabilities are highly receptive to disturbances in the flow, and can be amplified as they are advected downstream, resulting in transition to turbulence \cite{green_civil_2006}. The second mechanism is crossflow (CF) instabilities, which can arise on swept wings and are characterized by an inflection point in the transverse velocity profile. This inflection point causes the flow to be unstable, resulting in streamwise vortices that transition the flow to turbulence.

To delay this transition from laminar to turbulent flow, flow control methods can be implemented to push the transition location further downstream, increasing the regions of laminar flow \cite{green_civil_2003}. One promising passive approach is natural laminar flow (NLF), which relies on wing shaping to extend the laminar region. However, NLF design is generally restricted to modest sweep angles and Reynolds numbers \cite{lynde_computational_2017,lynde_preliminary_2019}.

At higher sweep angles and Reynolds numbers, active laminar flow control becomes necessary. HLFC extends natural laminar flow by adding boundary-layer suction to the aerodynamic shaping used in NLF. This combination enables significantly greater control over the boundary-layer profile, allowing transition to be pushed farther downstream. The technique is particularly suited to wings and empennages due to their large wetted areas and typically high sweep, while nacelles and other components may rely on other drag-reduction strategies \cite{johansen_prediction_1999}.


The concept of applying suction across the full upper and lower surfaces of a wing was experimentally explored by NACA in the Langley low-turbulence pressure tunnel \cite{Braslow1951}. Using a NACA 64A010 airfoil with a sintered bronze perforated surface, the study demonstrated that boundary-layer suction could maintain laminar flow at low speeds ($M \approx 0.3$) and moderate Reynolds numbers ($Re = 6 \times 10^6$). NASA later extended these investigations in the Langley Laminar-Flow-Control Experiment \cite{bobbitt_results_1992, harvey_design_1986, dagenhart_design_1978, berry_boundary-layer_1987}, applying suction over both upper and lower wing surfaces of a supercritical airfoil in unswept and swept configurations. The study validated transition prediction methods and compared slotted and perforated suction surfaces, accounting for temperature gradients and surface porosity in the efficiency calculations. Results showed that suction could successfully delay transition up to 60\% of the chord, confirming the effectiveness of both suction surface types in moving the transition location downstream. Building on these experimental studies, Fisher and Fischer demonstrated active laminar flow control on a JetStar aircraft during the Leading-Edge Flight Test (LEFT) \cite{fisher_development_1987}. The aircraft employed both slotted and perforated suction surfaces, with additional measures such as Krueger flaps with integrated deicer nozzles to protect against insect, ice, and particulate contamination. Both suction configurations were successfully implemented, with slotted surfaces proving more effective at delaying transition, while perforated surfaces were more prone to clogging.

Numerical studies of HLFC have also been undertaken. Sudhi et al.\ applied active laminar flow control to two-dimensional airfoils using XFOIL coupled with the $e^N$ transition prediction method \cite{Sudhi2023-yf}. Drag-minimization optimizations were performed with and without suction, treating the onset of suction as a design variable. Optimized suction distributions applied upstream of the natural transition location delayed transition to 80\% of the chord and reduced drag by 30\% relative to the no-suction case. Extending this methodology, Sudhi et al.\ explored HLFC on transonic, infinite swept wings at high Reynolds numbers \cite{Sudhi2023-yf}. By jointly optimizing wing shape and suction distribution using a multi-objective genetic algorithm, HLFC configurations achieved up to 25\% drag reduction over NLF designs at $Re = 30 \times 10^6$ and sweep angle $22.5^\circ$, outperforming fully turbulent designs by 27\%. Prasannakumar et al.\ \cite{Prasannakumar_Sudhi_Seitz_Badrya_2024} applied suction to a finite wing swept at $16^\circ$ at the operating conditions of a hybrid-electric propulsion aircraft. The aircraft has a cruise Mach number of 0.71, $C_L = 0.5$ and a range of 4600 km. The objective of this work was to maximize the laminar region while minimizing the suction and pressure drag through the control of suction panels at different spanwise stations at the leading edge. The results of this study indicate that the transition point is able to be moved to 60\% chord with considerations for a battery powering the suction system to maximize the net drag reduction.

Pascual and Zingg \cite{Pascual2025-bx} implemented a suction-based aerodynamic optimization framework for HLFC, coupling a suction boundary condition with the SA-sLM2015cc transition model \cite{piotrowski_smooth_2021, piotrowski_compressibility_2023} in a Reynolds-averaged Navier–Stokes (RANS) solver. Strategically placing multiple suction locations yielded significant drag reductions. On an NLF-optimized RAE2822 airfoil with operating conditions: $M=0.6$, $Re = 10 \times 10^6$, $C_L = 0.42$, suction applied upstream of the transition location reduced drag by 6\% at a suction speed of $0.1\% \: U_\infty$. For infinite swept wings with operating conditions: $M = 0.78$, $Re = 15 \times 10^6$, $C_L = 0.492$, $\Lambda = 25^\circ$, aerodynamic shaping alone achieved 5\% drag reduction, while applying suction upstream at a suction speed of $0.1\% \: U_\infty$, resulted in 8\% drag reduction, validating the flow physics of suction and the effect of suction on laminar-turbulent transition.

The objective of the current work is to develop an effective methodology to perform aerodynamic shape optimization in the presence of laminar-turbulent transition and suction, and to use the optimization methodology developed to study optimal approaches to drag reduction via HLFC and to quantify the performance benefits achievable. To validate the suction boundary condition implementation, several benchmark cases are examined, including flat-plate flow, airfoil experiments, and high-speed swept-wing configurations.

Section \ref{sec:methodology} outlines the flow solver, transition-prediction model, and aerodynamic shape optimization framework, including the modeling of active laminar flow control. Section \ref{sec:validation} presents the validation cases performed on flat plates and airfoils. Section \ref{sec:results} presents the results of lift-constrained drag minimization of two-dimensional airfoils with and without sweep and finite wings, incorporating suction in various ways. Finally, conclusions and future work are presented in Section \ref{sec:conclusion}.

\section{Methodology} \label{sec:methodology}

Aerodynamic shape optimization is carried out using Jetstream, the high-fidelity optimization framework developed at the University of Toronto Institute for Aerospace Studies. Jetstream comprises five core components: a Newton-Krylov-Schur flow solver for the Reynolds-Averaged Navier–Stokes (RANS) equations \cite{osusky_parallel_2008, osusky_parallel_2010}; a transition-prediction model based on empirical local correlations, coupled with the one-equation Spalart–Allmaras (SA) turbulence model \cite{piotrowski_investigation_2019, piotrowski_smooth_2021}; a unified geometry parameterization and mesh deformation scheme based on linear elasticity \cite{hicken_aerodynamic_2010}; the SNOPT gradient-based optimizer used alongside a discrete-adjoint gradient method \cite{gill_snopt_2002}; and a combination of free-form deformation (FFD) and axial deformation for geometry control \cite{gagnon_two-level_2015}. The framework has been extended to support active boundary-layer suction through the implementation of a suction boundary condition.

The flow solver, Diablo \cite{osusky_parallel_2008, osusky_parallel_2010}, is a multiblock, parallel, implicit RANS solver that employs second-order summation-by-parts (SBP) operators for spatial discretization and uses simultaneous approximation terms (SATs) \cite{delrey_fernandez_simultaneous_2018} to enforce boundary and block-interface conditions.

Geometry parametrization and mesh deformation follow the approach of Hicken and Zingg \cite{hicken_aerodynamic_2010}, in which each computational block is associated with a B-spline volume whose surface control points provide a low-dimensional representation of the geometry. Mesh motion is propagated to the interior using a linear-elasticity formulation. Geometry modification uses a hybrid strategy combining free-form and axial deformation methods \cite{gagnon_two-level_2015}. Free-form deformation (FFD) volumes govern section shape, chord, and twist through B-spline control points, while axial B-spline curves control sweep, span, and dihedral. For the infinite swept-wing cases considered here, axial-curve deformation is unnecessary but is included for completeness and applied in the finite-wing optimizations.

Gradient-based optimization is performed using SNOPT \cite{gill_snopt_2002}, which implements a sequential quadratic programming (SQP) framework capable of handling linear and nonlinear constraints. Gradients are supplied via the discrete-adjoint method \cite{Jameson1998-sf, Osusky2015-qh}.

\subsection{Transition Prediction Model} \label{sec:transition-prediction}
The flow solver incorporates boundary-layer transition prediction using the SA-sLM2015cc model developed by Piotrowski and Zingg \cite{piotrowski_smooth_2021, piotrowski_compressibility_2023}. This model builds on the $\gamma\text{-}Re_{\theta t}$ framework; an empirical, correlation-based approach introduced by Langtry and Menter \cite{langtry_correlation-based_2009} to model Tollmien–Schlichting (TS) and crossflow (CF) instabilities that includes the $\gamma\text{-}Re_{\theta t}$ model includes two transport equations: one for $\gamma$, the intermittency function, and another for $Re_{\theta t}$, the momentum-thickness Reynolds number. The SA-sLM2015 model \cite{piotrowski_smooth_2021} couples this transition framework with the one-equation Spalart–Allmaras turbulence model \cite{spalart_one-equation_1994}, and replaces non-smooth functions in the original formulation with smooth alternatives to ensure a continuous and differentiable design space. Piotrowski and Zingg later enhanced the model by incorporating compressibility corrections for both TS and CF instabilities, leading to the current version, SA-sLM2015cc \cite{piotrowski_compressibility_2023}.

\subsection{Suction Boundary Condition} \label{sec:suction-bc}

In this work, active boundary-layer suction is incorporated into the optimization framework through the definition of a suction velocity at the surface. The design process accounts for viscous effects, including the additional drag and pump power associated with suction, by introducing appropriate penalty terms in the objective function. The porosity of the surface is also considered, allowing for a physically consistent representation of the suction mechanism and its impact on the flow.

\subsubsection{Suction Drag Considerations}

The suction boundary condition formulation accounts for the drag effects introduced by active boundary-layer control. The suction drag represents the additional skin-friction component arising from the extraction of low-momentum fluid near the surface, which alters the boundary-layer velocity profile and increases the local wall shear stress. Following Fehrs \cite{Fehrs2020-qg}, the suction drag coefficient over a unit surface is expressed as:
\begin{equation}
\begin{aligned}
C_{D,\text{suc}} &= \frac{2 \cdot D_{\text{suc}}}{\rho_\infty U_\infty^2 	bl} = -2 \cdot \frac{v_w}{U_\infty}  = 2 \cdot C_Q.
\label{eq:suction-drag}
\end{aligned}
\end{equation}

\noindent This formulation relates the prescribed suction velocity to the additional skin-friction drag.

\subsubsection{Suction Pump Power Considerations}

Additionally, active suction requires energy to sustain the prescribed mass flux through the porous surface, which can be expressed as an equivalent power drag coefficient. Following the formulation of Prasannakumar et al.\ \cite{prasannakumar_design_2022} and assuming the pump efficiency, $\eta_p = 1$, the ideal suction pump power penalty is given by:

\begin{equation}
C_{D,\text{pump}} = \left( \frac{U_e}{U_\infty} \right)^2 \left( \frac{v_{\text{suc}}}{U_\infty} \right).
\label{eq:suction-pump-power}
\end{equation}

\noindent This expression captures the energy cost of suction in a compact form, allowing it to be directly incorporated into aerodynamic optimization as an equivalent drag contribution.

\section{Model Validation}
\label{sec:validation}
This section presents the validation of the developed suction boundary condition. First, two validation cases are discussed: boundary-layer suction applied to a flat plate and to a NACA 64A010 airfoil, demonstrating agreement with both computational and experimental reference data.

\subsection{Boundary-layer Suction Applied to a Flat Plate}
Fehrs implemented a no-slip wall boundary condition using an effusion mass flux boundary condition in the DLR TAU-Code \cite{Fehrs2020-qg} and validated it analytically. 	The operating conditions consist of $U_\infty = 40$ m/s, $Re = 6.62 \times 10^6$ with the suction coefficient and suction velocity listed in Table \ref{tab:Fehrs_operating_conditions}. The results in Figure \ref{fig:fehrs_flat_plate} show the velocity profiles with respect to the similarity variable $\eta$, which is defined as:
\[
\eta = y \sqrt{\frac{U_\infty}{2 \nu x}}.
\]
The present computational results show excellent agreement with the reference data.

\addtocounter{table}{-1}
\begin{table}[!t]
\centering
\caption{Inflow conditions for flat plate with boundary-layer suction}
\label{tab:Fehrs_operating_conditions}
\begin{tabular}{ccccc}
\toprule
Case & $C_Q$ & $v_w$ (m/s)\\
\midrule 
1 & $8.33 \times 10^{-4}$ & - 0.03342 \\
2 & $4.16 \times 10^{-3}$ & - 0.16712 \\
\bottomrule
\end{tabular}
\end{table}

\begin{figure}[!t]
\centering
\includegraphics[height = 8cm, width = \textwidth, keepaspectratio]{fehrs_flat_plate.pdf}
\caption{Boundary-layer velocity profiles with two different suction coefficients produced using the SA-sLM2015cc transition model validated against DLR TAU-Code computational results (shown with symbols).}
\label{fig:fehrs_flat_plate}
\end{figure}

\pagebreak

\subsection{Flow over an Airfoil with Prescribed Suction} \label{sub:workshop-infinite-swept}
Braslow et al.\ conducted experiments using a NACA 64A010 airfoil with a sintered bronze surface through which suction was applied \cite{Braslow1951}. Suction was applied across the entirety of the upper and lower surfaces, aside from a small portion at the leading edge at approximately $5\% \: x/c$. The wind tunnel was operated at a speed of $M \approx 0.3$ and $Re = 6 \times 10^6$ and the airfoil was fixed at $\alpha = 0^\circ$. The computational mesh used for this validation case is a 270 $\times$ 182 O-grid.
	
The experiment measured the boundary-layer profiles at 83\% of the chord and found that the flow was able to remain laminar. The results of this validation case are shown in Figure \ref{fig:Braslow_results}, with the computational results showing a laminar boundary-layer profile in agreement with the experimental results. 

\begin{figure}[!t]
\centering
\includegraphics[height=8cm, width = \textwidth, keepaspectratio]{braslow.pdf}
\caption{Boundary-layer velocity profiles at $\mathbf{83\%}$ chord comparing results obtained with the SA-sLM2015cc transition model to the NACA 64A010 wind tunnel experiment.}
\label{fig:Braslow_results}
\end{figure}

\begin{figure}[!t]
\centering
\includegraphics[height=7cm, width = \textwidth, keepaspectratio]{schwartzberg_extents.pdf}
\caption{Extent of laminar flow on the upper surface of a NACA 64A010 airfoil. Suction is applied across the entirety of the upper surface, except for the first 5\% chord with various suction coefficients used.}
\label{fig:Schwartzberg_results}
\end{figure}

\begin{figure}[!t]
\centering
\includegraphics[height=7cm, width = \textwidth, keepaspectratio]{bl_profiles.pdf}
\caption{Boundary-layer profiles for a NACA 64A010 airfoil. The profiles are taken on the upper surface taken at 75\% chord comparing experimental results (shown with symbols) with the present work for various suction coefficients.}
\label{fig:bl_profiles}
\end{figure}

With the same experimental operating conditions, geometry and suction application, Schwartzberg et al.\ captured boundary-layer profiles for several suction coefficients and analyzed the effect this variation had on the transition front \cite{Schwartzberg1952}. Fehrs also used the DLR-TAU code to replicate this experiment \cite{Fehrs2020-qg}. To compare against experimental data, the transition location for $C_Q=0$ must be matched, and to account for the surface roughness, Fehrs used a turbulence intensity of $Tu=0.6\%$. The cases run in the present work use a turbulence intensity of $Tu=0.63\%$. The results comparing the computational and experimental transition locations are plotted in Figure \ref{fig:Schwartzberg_results} with the present work demonstrating close agreement with both the experimental and the computational results. Boundary-layer profiles were captured as part of this experiment and are shown in Figure \ref{fig:bl_profiles}, comparing the experimental results with the present work. When a lower suction velocity of $C_Q = 0.00076$ is used, the profile has a thicker profile, which is more characteristic of a turbulent boundary-layer profile. When suction is increased to a velocity of $C_Q = 0.00310$, a thinner profile is obtained. These results show close agreement with the experimental values, providing validation for the current methodology.

\section{Optimization Results} \label{sec:results}
The methodology is applied to aerodynamic optimization studies involving infinite swept wings, where suction is applied to optimized geometries through discrete slot implementations. This study aims to quantify the aerodynamic performance improvements achievable through the integration of active laminar flow control into the design process.

\subsection{Suction Applied to an Optimized Geometry at Airbus A340 Operating Conditions}
\label{sub:a340-infinite-swept}
The following cases were completed using the RAE2822 airfoil as the initial geometry. At estimated Airbus A340 conditions consisting of $M = 0.82$, $Re = 45.34 \times 10^6$, $C_L = 0.512$, $\Lambda = 30^\circ$. The computational meshes used for the infinite swept wing cases are 360 $\times$ 120 O-grids. This geometry has 11 axial free-form deformation curves for each surface. Suction is applied at a fixed location on a geometry obtained from a lift-constrained drag minimization without suction, subject to an area constraint and minimum thickness constraints. The NLF optimization problem is summarized as:
\begin{align*}
	\min_{\textbf{X}} \quad \quad &C_D(\textbf{X}) \quad \quad \\
	\text{s.t.} \quad \quad
		&C_L =  0.512 \yesnumber \label{eq:a340_opt_problem} \\
		&A \geq A_{\text{init}} \\
		&t/c \geq 0.15 (t/c)_{\text{init}},
\end{align*}
\noindent where \textbf{X} represents the design variable vector, including the angle of attack and the FFD control points, $C_L$ and $C_D$ the lift and drag coefficients, respectively, $A$ the cross-sectional area, and $t/c$ the thickness-to-chord ratio for each FFD control point pair. The `init' subscript indicates the value of the quantity from the baseline geometry. Previous results by Pascual and Zingg \cite{Pascual2025-bx} investigated lower sweep angle and Reynolds number combinations. The present work shifts focus to conditions where aerodynamic shaping offers limited benefits, increasing reliance on active suction to delay transition and reduce drag. The presented cases account for the suction drag penalty defined in Equation \ref{eq:suction-drag}, with suction being applied over the entire leading edge up until the upper and lower surfaces' respective transition locations on the optimized geometry.

Results for the initial geometry analysed with transition prediction, a fully-turbulent optimized geometry for comparison, the NLF optimized geometry, and three different suction coefficients applied to the NLF optimied geometry are shown in Table \ref{tab:a340-results}. Figure \ref{fig:a340-cpcf} shows the pressure coefficient plot for the baseline, NLF-optimized, and suction configurations. As seen in the table, the NLF optimization provides an 11.8\% reduction in drag, but the optimizer struggles to significantly delay transition on both surfaces, especially on the lower surface. 

When suction is introduced, the aerodynamic performance improves further. Even the lowest suction level tested ($C_Q = 0.0001$), provides a drag reduction of 10.9\% compared to the turbulent optimized case. Increasing the suction coefficient consistently enhances the delay of transition, with $C_Q = 0.0010$ producing a 20.1\% drag reduction.

Figure \ref{fig:a340-transition-plot} illustrates the influence of suction on the transition location for both surfaces. As expected, higher suction velocities push the transition point further downstream. However, the spacing between transition locations decreases with increasing $C_Q$, indicating diminishing returns in transition delay at higher suction levels. Overall, while all suction levels produce substantial improvements relative to the NLF-optimized case, the marginal benefit decreases as the suction intensity is increased, suggesting that there is an optimal value for the suction coefficient.

\begin{table}[!t]
\centering
\caption{Summary of results for the Airbus A340 case with suction applied along the entire leading edge, extending to 9\% of the chord on the upper surface and 2.5\% on the lower surface. Percent drag reduction is shown compared to the turbulent optimized case.}
\label{tab:a340-results}
\begin{tabular}{ccccc}
\toprule
\multirow{2}{*}{Configuration} & \multirow{2}{*}{$C_D$ (counts)} & \multirow{2}{*}{L/D} & \multicolumn{2}{c}{Transition Location (x/c)} \\
& & & Upper Surface     & Lower Surface \\ 
\midrule 
Baseline  &  70.8 &  72.4 & 4.4\% & 2.2\% \\
Turbulent Optimized  &  66.5 &  77.1 & - & - \\
NLF Optimized & 62.4 (-5.71\%) & 82.0 & 9.23\% & 2.6\% \\
\midrule
\textbf{Suction Coefficient} &  & & &\\
$C_Q = 0.0001$ & 59.2 (-10.9\%) & 87.7  & 15.0\% & 3.8\% \\
$C_Q = 0.0005$ & 57.3 (-13.8\%) & 92.2  & 19.6\% & 6.4\% \\
$C_Q = 0.0010$ & 56.6 (-14.9\%) & 94.4 & 21.0\% & 8.3\% \\
\bottomrule
\end{tabular}
\end{table}

\begin{figure}[!t]
\centering
\includegraphics[height = 7cm, width = \textwidth, keepaspectratio]{a340-cpcf1.pdf}
\caption{Pressure coefficient plot at Airbus A340 operating conditions with suction applied along the entire leading edge, extending to 9\% of the chord on the upper surface and 2.5\% on the lower surface.}
\label{fig:a340-cpcf}
\end{figure}

\begin{figure}[!t]
\centering
\includegraphics[height = 7cm, width = \textwidth, keepaspectratio]{a340-transition-plot1.pdf}
\caption{Transition location on the upper and lower surface at Airbus A340 operating conditions for various suction velocities when applied to the optimized geometry.}
\label{fig:a340-transition-plot}
\end{figure}

\subsubsection{Application of Suction at the Updated Transition Location}
\label{subsub:a340-updated}
Following the analysis in the previous section with suction applied to the leading edge, suction was then applied beginning approximately 3\% upstream of each newly identified transition point. The resulting suction extents and corresponding drag levels are summarized in Table \ref{tab:a340-results-2} and the pressure distributions shown in Fig. \ref{fig:a340-cpcf-2}. Fig. \ref{fig:a340-transition-plot-2} illustrates that transition on the upper surface is displaced downstream for all suction levels, with larger shifts occurring at higher $C_Q$. This delay in transition produces a corresponding reduction in drag, decreasing from 58.46 counts at $C_Q = 0.0001$ to 53.79 counts at $C_Q = 0.0010$.

\begin{table}[!t]
\centering
\caption{Summary of results for the Airbus A340 case with suction applied along the entire leading edge, extending to 9\% of the chord on the upper surface and 2.5\% on the lower surface. The transition location was analyzed and suction was then applied with an extent 3\% upstream of updated transition location for the respective suction coefficient.}
\label{tab:a340-results-2}
\begin{tabular}{cccccc}
\toprule
\multirow{2}{*}{Suction Coefficient} & \multicolumn{2}{c}{Additional Suction Location (x/c)} & \multirow{2}{*}{$C_D$ (counts)} & \multicolumn{2}{c}{Transition Location (x/c)} \\
& Upper Surface     & Lower Surface    & & Upper Surface & Lower Surface  \\
\midrule
$C_Q = 0.0001$ & 12.0\% -- 15.0\% & 2.5\% -- 3.7\% &  58.46  & 15.6\% & 4.0\% \\
$C_Q = 0.0005$ & 16.5\% -- 19.5\% & 3.5\% -- 6.5\% & 55.71  & 21.3\% & 7.9\% \\
$C_Q = 0.0010$ & 18.0\% -- 21.0\% & 5.3\% -- 8.3\% & 53.79  & 21.0\% & 15.3\% \\
\bottomrule
\end{tabular}
\end{table}

\begin{figure}[!t]
\centering
\includegraphics[height = 7cm, width = \textwidth, keepaspectratio]{a340-cpcf2.pdf}
\caption{Pressure coefficient plot at Airbus A340 operating conditions with suction applied along the entire leading edge, extending to 9\% of the chord on the upper surface and 2.5\% on the lower surface.}
\label{fig:a340-cpcf-2}
\end{figure}

\begin{figure}[!t]
\centering
\includegraphics[height = 7cm, width = \textwidth, keepaspectratio]{a340-transition-plot2.pdf}
\caption{Transition location on the upper and lower surface at Airbus A340 operating conditions for various suction velocities when applied to the NLF optimized geometry.}
\label{fig:a340-transition-plot-2}
\end{figure}

%\subsection{Suction Applied to an Optimized Geometry at Airbus A350 Operating Conditions}
%
%This geometry is operated at estimated Airbus A350 conditions consisting of $M = 0.85$, $Re = 38.01 \times 10^6$, $C_L = 0.478$, $\Lambda = 31.9^\circ$. The results for the application of suction to both the upper and lower surfaces upstream of the transition location are shown in Table \ref{tab:a350-results}, with suction applied along the entire leading edge, extending to 9\% of the chord on the upper surface and 2\% on the lower surface. The NLF-optimized shape provides a large improvement, reducing the drag to 55.09 counts (40.58\% drag reduction). When active suction is introduced, additional drag reduction is achieved beyond that obtained through shaping alone. The lowest suction coefficient tested ($C_Q = 0.0001$) decreases the drag to 52.68 counts, corresponding to a 43.18\% reduction relative to the baseline. As the suction level is increased, the drag continues to decrease, reaching 50.56 counts for $C_Q = 0.0010$, although the marginal improvement diminishes between the two highest suction levels. Similar to the case	 previously shown, the results demonstrate that the A350 configuration benefits significantly from both aerodynamic shaping and active suction, with the combined approach providing the largest improvements with diminishing returns at higher suction intensities.
%
%\begin{table}[!t]
%\centering
%\caption{Summary of results for the Airbus A350 case with suction applied along the entire leading edge, extending to 9\% of the chord on the upper surface and 2\% on the lower surface.}
%\label{tab:a350-results}
%\begin{tabular}{cccc}
%\toprule
%Configuration & $C_D$ (counts) & $L/D$ \\
%\midrule 
%Baseline  & 92.72 & 51.55 \\
%Turbulent Optimized  & 62.40 (-32.70\%) & 76.63 \\
%NLF Optimized & 55.09 (-40.58\%) & 88.51 \\
%\midrule
%$C_Q = 0.0001$ & 52.68 (-43.18\%) & 91.85 \\
%$C_Q = 0.0005$ & 50.89 (-45.11\%) & 96.39\\
%$C_Q = 0.0010$ & 50.56 (-45.47\%) & 97.43 \\
%\bottomrule
%\end{tabular}
%\end{table}
%
%\begin{figure}[!t]
%\centering
%\includegraphics[height = 7cm, width = \textwidth, keepaspectratio]{a350-cpcf1.pdf}
%\caption{Pressure coefficient plot at Airbus A350 operating conditions with suction applied along the entire leading edge, extending to 9\% of the chord on the upper surface and 2\% on the lower surface.}
%\label{fig:a350-cpcf}
%\end{figure}
%
%\begin{figure}[!t]
%\centering
%\includegraphics[height = 7cm, width = \textwidth, keepaspectratio]{a350-transition-plot1.pdf}
%\caption{Transition location on the upper and lower surface at Airbus A350 operating conditions for various suction velocities when applied to the NLF optimized geometry.}
%\label{fig:a350-transition-plot}
%\end{figure}
%
%\subsubsection{Application of Suction at the Updated Transition Location}
%Similar to Section \ref{subsub:a340-updated}, suction was next applied to the optimized A350 geometry beginning approximately 3\% upstream of each updated transition location identified from the baseline simulations with leading-edge suction. For each suction coefficient, the flow was recomputed and the transition location re-evaluated, after which the suction region was shifted accordingly.
%
%The resulting suction extents and drag levels are presented in Table \ref{tab:a350-results-2} with the pressure distributions shown in Fig. \ref{fig:a350-cpcf-2}. As the suction coefficient increases from $C_Q = 0.0005$ to $C_Q = 0.0010$, the suction region moves farther aft on both the upper and lower surfaces, consistent with the increased ability of stronger suction to stabilize the boundary layer and delay transition. 
%
%Overall, the transition location on the upper surface is shifted downstream for both suction levels, with the stronger suction case producing the larger displacement. This results in a modest reduction in drag, from 49.45 counts at $C_Q = 0.0005$ to 49.11 counts at $C_Q = 0.0010$, demonstrating that suction applied near the updated transition location remains effective even on the optimized A350 geometry.
%
%\begin{table}[!t]
%\centering
%\caption{Summary of results for the Airbus A350 case with suction applied along the entire leading edge, extending to 9\% of the chord on the upper surface and 2\% on the lower surface. The transition location was analyzed and suction was then applied with an extent 3\% upstream of updated transition location for the respective suction coefficient}
%\label{tab:a350-results-2}
%\begin{tabular}{cccc}
%\toprule
%\multirow{2}{*}{Suction Coefficient} & \multicolumn{2}{c}{Suction Location (x/c)} & \multirow{2}{*}{$C_D$ (counts)} \\
%                                     & Upper Surface     & Lower Surface    &                              \\
%\midrule
%$C_Q = 0.0005$ & 13.1\% -- 16.1\% & 2.0\% -- 5.0\% & 49.45 \\
%$C_Q = 0.0010$ & 15.1\% -- 18.1\% & 3.4\% -- 6.4\% & 49.11 \\
%\bottomrule
%\end{tabular}
%\end{table}
%
%\begin{figure}[!t]
%\centering
%\includegraphics[height = 7cm, width = \textwidth, keepaspectratio]{a350-cpcf2.pdf}
%\caption{Pressure coefficient plot at Airbus A350 operating conditions with suction applied along the entire leading edge, extending to 9\% of the chord on the upper surface and 2\% on the lower surface.}
%\label{fig:a350-cpcf-2}
%\end{figure}
%
%\begin{figure}[!t]
%\centering
%\includegraphics[height = 7cm, width = \textwidth, keepaspectratio]{a350-transition-plot2.pdf}
%\caption{Transition location on the upper and lower surface at Airbus A350 operating conditions for various suction velocities when applied to the optimized geometry.}
%\label{fig:a350-transition-plot-2}
%\end{figure}
%
%%\subsection{Suction Applied to an Optimized Geometry at Boeing 747 Operating Conditions}
%%
%%This geometry is operated at Boeing 747 conditions consisting of $M = 0.854$, $Re = 55.14 \times 10^6$, $C_L = 0.457$, $\Lambda = 37.5^\circ$. The computational meshes used for this infinite swept wing are 360 $\times$ 120 O-grids. This geometry has 11 axial free-form deformation curves for each surface.
%%
%%The results for the application of suction to both the upper and lower surfaces upstream of the transition location are shown in Table \ref{tab:b747-results}. Figure \ref{fig:b747-cpcf} shows that the Pressure coefficient plot for the baseline and optimized geometry, and one with active suction (applied at $0.1\% \: U_\infty$) for clarity. For this preliminary optimization, the optimizer is struggling to delay transition on both surfaces. When active suction is applied, it can be seen that the upper surface's transition location is moved downstream for the case shown, resulting in a 10.6\% reduction in drag. The effect of $C_Q$ on the transition location is shown in Figure \ref{fig:b747-transition-plot}, which depicts both the upper and lower surfaces for the suction velocities used in this test. As expected, the transition location is pushed further downstream when a higher suction velocity is used, with substantial benefits compared to the preliminary NLF-optimized case shown with the blue markers.
%%
%%\begin{table}[!t]
%%\centering
%%\caption{Summary of results for infinite swept wing with suction. For the optimized geometry, the suction boundary condition is placed at 4\% of the chord with an extent of 1.5\% upstream.}
%%\label{tab:b747-results}
%%\begin{tabular}{cccc}
%%\toprule
%%Configuration & $C_D$ (counts) & $L/D$ \\
%%\midrule 
%%Baseline  & 73.88 & 68.22 \\
%%NLF Optimized & 68.11 (-7.80\%) & 73.16 \\
%%\midrule
%%$C_Q = 0.0001$ & 67.46 (-8.69\%) & 74.23 \\
%%$C_Q = 0.0005$ & 66.04 (-10.61\%) & 76.67\\
%%$C_Q = 0.0010$ & 64.14 (-13.18\%) & 80.36 \\
%%$C_Q = 0.0050$ & 64.14 (-13.18\%) & 80.36 \\
%%
%%\bottomrule
%%\end{tabular}
%%\end{table}
%%
%%\begin{figure}[!t]
%%\centering
%%\includegraphics[height = 10cm, width = \textwidth, keepaspectratio]{cpcf.pdf}
%%\caption{Pressure coefficient plot for cases involving the application of suction to both the upper and lower surface of an optimized geometry. Suction is applied at approximately 3\% of the wing chord on both surfaces.}
%%\label{fig:b747-cpcf}
%%\end{figure}
%%
%%\begin{figure}[!t]
%%\centering
%%\includegraphics[height = 10cm, width = \textwidth, keepaspectratio]{transition_plot.pdf}
%%\caption{Transition location on the upper and lower surface for various suction velocities when applied to the optimized geometry.}
%%\label{fig:b747-transition-plot}
%%\end{figure}
%
%\subsection{Suction Drag and Suction Pump Power Penalty Considerations}
%This section evaluates the penalty implications associated with the suction drag, defined in Equation \ref{eq:suction-drag}, and suction-pump requirements, defined in Equation \ref{eq:suction-pump-power}, providing insight into their influence on the overall performance of the HLFC system. The following cases were completed with the same geometry and operating conditions as Section \ref{sub:a340-infinite-swept} for the Airbus A340. The results in Table \ref{tab:penalty-considerations} show the application of suction over the leading edge of the wing penalties for suction drag, suction pump power, and a combination of both considered, for various suction coefficients. The results demonstrate that for suction coefficients less than $C_Q = 0.0001$, the amount of drag from both suction drag penalties is nearly negligible, with the combination contributing 0.29\%. When suction power is increased, as shown with $C_Q = 0.0005$ and $0.0010$, the suction drag has an expected increasing contribution to the drag. When a relatively high suction coefficient of $C_Q=0.0050$ is used, the suction applied becomes excessive, no longer providing stabilization but rather increasing drag and potentially triggering earlier turbulent transition.
%
%\begin{table}[!t]
%\centering
%\caption{Summary of results for infinite swept wing with suction at Airbus A340 operating conditions. Combinations of two different suction penalties are considered while varying the suction coefficient.}
%\label{tab:penalty-considerations}
%\begin{tabular}{cccc}
%\toprule
%Penalty & $C_D$ (counts) & $L/D$ \\
%\midrule
%$\mathbf{C_Q = 0.0001}$ & & \\
%No Penalty  & 59.07 &  87.84 \\
%Suction Drag & 59.17 (0.17\%) & 87.69 \\
%Pump Power & 59.14 (0.12\%) & 87.74 \\
%Combined & 59.24 (0.29\%) & 87.58 \\
%\midrule
%$\mathbf{C_Q = 0.0005}$ & & \\
%No Penalty  & 56.79 & 93.01 \\
%Suction Drag & 57.29 (0.88\%) & 92.18 \\
%Pump Power & 57.13 (0.60\%) & 92.44 \\
%Combined & 57.63 (1.48\%) & 91.62 \\
%\midrule
%$\mathbf{C_Q = 0.0010}$ & & \\
%No Penalty  & 55.55 & 96.09 \\
%Suction Drag & 56.55 (1.80\%) & 94.36 \\
%Pump Power & 56.23 (1.22\%) & 94.89 \\
%Combined & 57.24 (3.04\%) & 93.21 \\
%\midrule
%$\mathbf{C_Q = 0.0050}$ & & \\
%No Penalty  & 59.29 & 87.53 \\
%Suction Drag & 64.31 (8.47\%) & 80.63 \\
%Pump Power & 62.68 (5.72\%) & 82.62 \\
%Combined & 67.71 (14.20\%) & 76.35 \\
%\bottomrule
%\end{tabular}
%\end{table}
%
%\subsection{Suction Extent and Porosity Considerations}
%This section investigates the influence of varying suction extent on the transition front, while maintaining a constant $C_Q$, and also incorporates local wall-shear effects by prescribing a node-wise porosity factor that reflects the shear conditions at each surface location. These cases were completed using the RAE2822 airfoil as the initial geometry, with operating conditions of $M = 0.785$, $Re = 20.3 \times 10^6$, $C_L = 0.5$, $\Lambda = 25^\circ$. The computational meshes for these geometries are 300 x 122 O-grids. Suction is applied on the upper surface from 20\% -25\% chord with a suction coefficient of $C_Q = 0.0001$.
%
%\subsubsection{Effect of Suction Extent on Transition Location with a Constant Suction Coefficient}
%\label{subsub:suction-extent}
%The application of suction is shown in Figure \ref{fig:suction-extent}, where over a range of 15 nodes, the effect of decreasing the extent while maintaining $C_Q$ constant by increasing the local suction velocity is investigated. 
%
%\begin{table}[!t]
%\centering
%\caption{Summary of results for infinite swept wing while varying the suction extent upstream of the transition location and maintaining a constant suction coefficient.}
%\label{tab:suction-extent}
%\begin{tabular}{ccc}
%\toprule
%Configuration & $v_{\text{suc}}$ & $C_D$ (counts) \\
%\midrule 
%Baseline  & - & 60.25 \\
%\midrule
%\textbf{Suction Nodes} & & \\
%15/15 Nodes & $0.0100 \: U_\infty$ & 58.67 (-2.62\%)  \\
%14/15 Nodes & $0.0107 \: U_\infty$ & 58.70 (-2.57\%)  \\
%13/15 Nodes & $0.0115 \: U_\infty$ & 58.73 (-2.52\%)  \\
%12/15 Nodes & $0.0125 \: U_\infty$ & 58.75 (-2.49\%) \\
%11/15 Nodes & $0.0136 \: U_\infty$ & 58.79 (-2.42\%)  \\
%10/15 Nodes & $0.0150 \: U_\infty$ & 58.83 (-2.36\%)  \\
%9/15 Nodes & $0.0167 \: U_\infty$  & 58.87 (-2.29\%)  \\
%8/15 Nodes & $0.0188 \: U_\infty$  & 58.95 (-2.16\%)  \\
%\bottomrule
%\end{tabular}
%\end{table}
%
%\clearpage
%
%\subsubsection{Wall Porosity Considerations}
%This section investigates wall porosity considerations using a coefficient to allocate part of the drag over the specified region as suction drag and the rest as skin friction and pressure drag, treating it as a solid wall. This results in the overall drag over the suction region being defined as:
%
%\begin{equation}
%C_{\text{d,total}} = (1-p) \cdot (C_{\text{d,f}} + C_{\text{d,p}}) + p \cdot C_{\text{d,suc}},
%\label{eq:porosity}
%\end{equation}
%
%\noindent where $p$ is the porosity of the region and $C_{\text{d,suc}}$ is the suction drag defined in Equation \ref{eq:suction-drag}. The results for the application of suction in Table \ref{tab:wall-porosity-results} demonstrate that when suction is applied from 20\% -- 25\% chord on the upper surface, there is small change overall drag coefficient. The small variation in overall drag with porosity is expected, as the suction extent is small and located in an area where the boundary layer is already relatively thin. As a result, the contribution of suction drag to the total drag is modest, and the redistribution between suction and skin-friction drag has limited influence. For configurations where suction is distributed over a larger chordwise extent, the suction drag contribution would represent a larger proportion of the total drag.
%
%\begin{table}[!t]
%\centering
%\caption{Summary of results for infinite swept wing while varying the porosity of the surface over which suction is applied. A suction coefficient of $\mathbf{C_Q = 0.0001}$ is used.}
%\label{tab:wall-porosity-results}
%\begin{tabular}{cccc}
%\toprule
%Porosity & $C_{\text{d,total}}$ (counts) & L/D \\
%\midrule
%1.00 & 58.67 & 79.03 \\
%0.10 & 58.95 (0.47\%) & 78.65 \\
%0.20 & 58.92 (0.43\%) & 78.69 \\
%0.30 & 58.89 (0.37\%) & 78.74 \\
%0.40 & 58.86 (0.33\%) & 78.78 \\
%0.50 & 58.83 (0.27\%) & 78.82 \\
%\bottomrule
%\end{tabular}
%\end{table}

\section{Conclusions}
\label{sec:conclusion}
In this paper, we present a framework for aerodynamic shape optimization incorporating Hybrid Laminar Flow Control (HLFC) through active boundary-layer suction. The methodology was first validated against benchmark cases including flat-plate flow and NACA 64A010 airfoil experiments, demonstrating accurate prediction of laminar-to-turbulent transition and suction effects. Lift-constrained drag minimization was then performed for airfoils and infinite swept wings with fixed suction applications. For infinite swept wings at estimated Airbus A340 operating conditions, aerodynamic shaping combined with suction delayed crossflow-dominated transition, yielding 15\% drag reduction compared to what is achieved by NLF optimization. 

Future work will focus on extending this framework to three-dimensional, finite wings to capture realistic planform effects and full spanwise flow behavior. Optimization of suction power and placement will also be incorporated to identify the most effective combination of suction intensity and location for drag reduction while minimizing energy expenditure. These developments will enable more realistic assessments of HLFC performance for practical aircraft configurations and support the design of efficient suction-based laminar flow control systems.



\section*{Acknowledgements}
This work is partially funded by Bombardier, Transport Canada, the Natural Sciences and Engineering Research Council (NSERC), and the University of Toronto. All results in this paper were computed on the Niagara and Trillium supercomputers at the SciNet HPC Consortium, a part of the Digital Research Alliance of Canada.

\pagebreak 

\bibliography{MyLibrary}

\end{document}